\chapter{The Missing Baryon Problem}
	%  - Missing Baryon Problem
			%-  Stellar Baryons
			%- 	Cold Interstellar Medium
			%- 	Residual Lyman Alpha 
			%- 	OVI and BLA Absorbers
			%- 	Hot Gas in Clusters 

The Missing Baryon problem is one that arises when we try and make an account of the baryons at low redshift. At high redshift ($z>2$), the Lyman-$\alpha$ forest provides a good measure of the proportion of baryons, since at these redshifts the majority of the baryons in the universe are contained in diffuse gas. These analyses give a reported value 
$$\Omega_{baryon} \geq 0.035 $$ 

Observed light-element ratios and standard nucleosynthesis allows for direct computation of the expected baryon densities, which is in agreement with the above figure ***(Burles and Tytler 1998)***
$$\Omega_{baryon} = (0.019\pm 0.001)h^{-2} = 0.039 \pm 0.002 $$

The agreement between these two measures of baryon density, and the measurement obtained from the CMB lends confidence to the value obtained. 

\par However, at low redshifts, all analysis indicates that the summing over all observed contributions gives a value of 

$$\Omega_\star + \Omega_{HI} + \Omega_{H_2} + \Omega_{X-Ray,cl} \approx 0.0068 \leq 0.011 $$

This severe discrepency between measurements at high and low redshifts suggests that either the majority of the baryons at low redshifts are yet to be detected, or there are fundamental errors in numerous independant measures of the baryon density at high redshift. 

\section{Stellar Baryons}
The most obvious location to search for baryons are in the stellar populations of galaxies. At a broad level, we can imagine that there are two distinct stellar populations which can be considered to be found in high density galaxies; a class of old stars which exists in the spheroidal region of a galaxy, and a class of young stars in the disk region, as well as a third population existing in irregular galaxies.

\par Estimating the proportion of stellar baryons therfore becomes an exercise in galactic morphology and luminosity density function computation. *** Do I need to go through the explicit calculuation of star densities here?*** Perfoming this calculation gives mean mass density numbers for three classes of stars
\begin{align*}
\Omega_{\text{Spheroid Stars}} &= (0.00180^{+0.00121}_{-0.00085}) h^{-1} \\
\Omega_{\text{Disk Stars}} &= (0.00060^{+0.00030}_{-0.00024}) h^{-1} \\
\Omega_{\text{Stars in Irregular Galaxies}} &= (0.000048^{+0.000033}_{0.000026}) h^{-1}
\end{align*}

These numbers depend on the mass-to-light ratio for age estimation, and so in turn are dependant on the cosmological parameters in a complex way. Even if efforts were made to remove this dependancy by changing the methodology used to calculate the mass-to-light function, the necessity for the new methodology to hold consistent with other measurements would force the dependancy regardless ***(If dynamics were used toestimateM/L, the estimates of)would not depend onh,but in that casehB0.7 would be needed for dynamics toagree with the synthesis calculations. Either way, consis-tency holds forhB0.7 in a low-density universe.)***

\section{Cold Interstellar Medium}
\section{Lyman $\alpha$}
\section{OVI and BLA Absorbers}
\section{Hot Gas in Clusters}

