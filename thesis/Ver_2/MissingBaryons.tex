\chapter{The Missing Baryon Problem}
	%  - Missing Baryon Problem
			%-  Stellar Baryons
			%- 	Cold Interstellar Medium
			%- 	Residual Lyman Alpha 
			%- 	OVI and BLA Absorbers
			%- 	Hot Gas in Clusters 

The Missing Baryon problem is one that arises when we try and make an account of the baryons at low redshift. At high redshift ($z>2$), the Lyman-$\alpha$ forest provides a good measure of the proportion of baryons, since at these redshifts the majority of the baryons in the universe are contained in diffuse gas. These analyses give a reported value 
$$\Omega_{baryon} \geq 0.035 $$ 

Observed light-element ratios and standard nucleosynthesis allows for direct computation of the expected baryon densities, which is in agreement with the above figure \citep{1998sese.conf..113B}
$$\Omega_{baryon} = (0.019\pm 0.001)h^{-2} = 0.039 \pm 0.002 $$

The agreement between these two measures of baryon density, and the measurement obtained from the CMB lends confidence to the value obtained. 

\par However, at low redshifts, all analysis indicates that the summing over all observed contributions gives a value of 

$$\Omega_\star + \Omega_{HI} + \Omega_{H_2} + \Omega_{X-Ray,cl} \approx 0.0068 \leq 0.011 $$

This severe discrepency between measurements at high and low redshifts suggests that either the majority of the baryons at low redshifts are yet to be detected, or there are fundamental errors in numerous independant measures of the baryon density at high redshift. 

\section{Stellar Baryons}
The most obvious location to search for baryons are in the stellar populations of galaxies. At a broad level, we can imagine that there are two distinct stellar populations which can be considered to be found in high density galaxies; a class of old stars which exists in the spheroidal region of a galaxy, and a class of young stars in the disk region, as well as a third population existing in irregular galaxies.

\par Estimating the proportion of stellar baryons therfore becomes an exercise in galactic morphology and luminosity density function computation. There are three primary classes of stars which need to be considered, stars contained in the central disk of the galaxy, starts contained in the bulge (spheroid), and starts contained in irregular galaxies, which don't conform to simple catergorical definitions. Perfoming this calculation gives mean mass density numbers for these three classes of stars gives
\begin{align*}
\Omega_{\text{Spheroid Stars}} &= (0.00180^{+0.00121}_{-0.00085}) h^{-1} \\
\Omega_{\text{Disk Stars}} &= (0.00060^{+0.00030}_{-0.00024}) h^{-1} \\
\Omega_{\text{Stars in Irregular Galaxies}} &= (0.000048^{+0.000033}_{0.000026}) h^{-1}
\end{align*}

These numbers depend on the mass-to-light ratio for age estimation, and so in turn are dependant on the cosmological parameters in a complex way. Even if efforts were made to remove this dependancy by changing the methodology used to calculate the mass-to-light function, the necessity for the new methodology to hold consistent with other measurements would force the dependancy regardless.

\section{Cold Interstellar Medium}
We also know that some of the baryons in the local universe are stored in the Cold Interstellar Medium. The cold interstellar medium is a term for neutral and molecular gas, primarily consisting of unionised hydrogen gas (HI). The HI present in gas-rich galaxies is the best tracer for the neutral hydrogen mass content in the near universe, since it can be shown that very little neutral gas is present in objects outside the galaxy population in these regions. At redshift $z \approx 0$, there have not been any free HI clouds detected by radio surveys that have not yet subsequently been detected in optical bands \citep{1993ApJ...419..515R} , and so optical detection in galaxies can be considered to be adequate when taking a census of the total baryons contained in the cold interstellar medium. 
\par The HI content of a given galaxy can be calculated from the optical luminosity functions of given galaxy morphological types. For a given optical luminosity function $\phi^T(M)$, and a given HI mass content $M_{HI}^T(M)$, as functions of absolute magnitude $M$, the total HI mass content of a given galaxy morphological type $T$ can be found by computing an integral over all luminosities
$$M_{HI} = \int \phi^T(M) M_{HI}^T(M) dM $$

Adding all the contributions for all morphological types will therefore give an estimate of the total HI at $z=0$. At these distances, spiral galaxies are the primary contributors to the neutral hydrogen content, with $89\%$ of the hydrogen mass found in types later than S0 spiral galaxies. Initial estimates \citep{1993ApJ...419..515R} place the mass density of neutral hydrogen at $z=0$ at 

$$\Omega_{HI} = (2.5 \pm 0.6) \times 10^{-4} h_{75}^{-1} $$

Later surveys, such as the HI Parkes All-Sky Survey (HIPASS), refined this measurement further \citep{2003AJ....125.2842Z}. HIPASS is a blind survey of the southern sky south consisting of approximately 7000 galaxies. The much higher number of galaxies in this survey necessitated a different method of computing the neutral hydrogen fraction. Taking a statistical approach, and assuming that the survey is inherently not detecting neutral mass fraction below a certain magnitude, the probability that a galaxy with a given HI mass is 
$$ p(M_{HI,i}|D) = \frac{\Theta(M_{HI,i})}{\int_{M_{HI,\lim(D_i)}}^\infty \Theta(M_{HI}) dM_{HI}} $$
, where $M_{HI,\lim(D_1)}$ is the minimum detectable HI mass at some distance $D_i$. This essentially gives the fraction of galaxies in the survey with a given HI mass sufficient to be detected. The parent distribution $\Theta$ can then be maximised by finding which product of probabilities is also maximal. The statistical nature of this method requires accounting for various forms of bias, such as selection bias in the survey, the Eddington effect, self-absorption of the neutral hydrogen, and cosmic variance. This methodology gives a measure of the mass density as 
$$\Omega_{HI} = (3.8 \pm 0.6) \times 10^{-4} h_{75}^{-1} $$
\section{Photoionised Lyman $\alpha$}
At higher redshifts, the entirety of the baryon content in the universe can essentially be found in large quanties of gas that have not yet collapsed into galaxies, which shows up very clearly in the Lyman $\alpha$ (Ly$\alpha$) forest. At low redshifts, we can search for the residual Ly$\alpha$ signal using HI absorber frequencies in distant quasars. By making the assumption that these absorbers are isothermal spheres, and choosing a given impact parameter, the total cloud mass can be essentially inferred from the HI column density. 

\par One model for estimating the contribution of the local Lyman $\alpha$ forest is outlined by \cite{2000ApJ...544..150P}. From big bang nucleosynthesis, the total baryons contributed from the low-$z$ Ly$\alpha$ is given by 
$$\Omega_{Ly\alpha} = \int_{N_{min}}^{N_{max}} \frac{ \phi_0(N_{HI},p) M_{cl}(N_{HI}, p, J_0)}{\rho_{crit}} dN_{HI} $$
, where $M_{cl}(N_{HI}, p, J_0)$ is the mass of an individual cloud , $\phi_0(N_{HI},p)$ is space density of the clouds, $N_{HI}$ is the column density of the clouds, $J_0$ is the specific intensity of the metagalactic ionizing radiation field ,$\rho_{crit} = 3 H_0^2/8 \pi G $ is the critical density at the present day which is necessary to halt the expansion of the universe.
\par Making more assumptions about the compostion of the spheres, and that the gas is in photoionising equilibrium, the impact parameter, this reduces the integral to
$$\Omega_{Ly\alpha} = (0.008 \pm 0.001) \left[ J_{-23} p_100 \left(\frac{ 4.8}{ \alpha_s + 3} \right) \right]^{1/2} h_{70}^{-1} $$
, where $\alpha_s$ is the spectral index of the radiation field, $J_{-23}= J_0/10^{-23}$, $p_{100} = p/(100 kpc)$ and $h_{70} = H_0/(70 Mpc)$ . This corresponds to approximately $20\%$ of the baryons, but it inherently makes a number of rather significant assumptions, which are highly dependant on the number of Lyman $\alpha$ absorbers detected at low redshifts. 
\par Further accounting for the fact that the clouds are gravitationally bound, and that their densities are typical over a characteristic Jeans length allows for further refinement of this number \citep{2001ApJ...559..507S}. Doing so yields a measure of the number of baryons in the residual Lyman $\alpha$ forest of $\Omega_{Ly\alpha} = 29 \% \pm 4\%$ \citep{ 2004ApJS..152...29P,2008ApJ...679..194D}  
\subsection{OVI and BLA Absorbers}

Because gas at low redshift exists at a large range of temperatures, and at a higher metalicity that the same gas at higher redshifts, distant quasars will exhibit absorption from higher Lyman alpha lines, such as oxygen (OVI), and broad Lyman $\alpha$ lines. The OVI absorption line probes gas in temperature ranges from $10^5$ - $10^6$ K, which has been shock heated as a result of gravitational instability during structure formation. Gas hotter than this is only sensitive to very weak absorption lines from higher ions, such as O VII, O VIII, Ne X, N VI, and N VII, which are only detectable in weak X-Rays \citep{2005ApJ...624..555D}. 
\par Broad Lyman $\alpha$ also act as a tracer of the gas in these temperature ranges. Theory suggests that the ionisation equilibrium of gas should contain a very small portion of neutral gas, typically $f \sim 10^{-5} - 10^{-6}$, so there should be some Ly $\alpha$ emission, thermally broadened by the intervening gas \citep{2006A&A...445..827R}. It is not entirely clear if the OVI and BLA absorbers trace the same underlying gas phase, so they have been included here together. 
\par The method of calculating these numbers is a comparatively simply integral, over the number of absorbers per redshift bin, $dN_{OVI,BLA}/dz$. This does make several assumptions which are sensitive to the visibility, as well as the number of absorbers in a given survey. \par \cite{2005ApJ...624..555D} found the proportion of matter contained in OVI absorbers to be approximately $5\%$, extrapolated from $\sim 50$ absorbers. However, this number is highly dependant on model-dependant spread of oxygen/hydrogen metalicities, the redshift range of the survey, and OVI ionisation fractions, leading to large errors on this measurement. 
\par \cite{2006A&A...445..827R} found the proportion from BLAs to be $\sim 15 \% - 150 \%$ of the total baryon fraction, from a sample of between 20 and 50 sources. The reason for the large uncertainty is the inherent uncertainty in the detection of BLA sources. In fact, the unreasonably high measurement of the baryon fraction says that this method is inherently counting sources which are not actually good tracers of the free baryons. There is also some confusion regarding sources which appear as both OVI and BLA absorbers, since the connection between the two types are not clear. 

\section{Summary}
These methods all carry some sources of error with them, some so large that they call into question their validity entirely. These errors stem from the underlying assumptions assosciated with their method of calculation, as well as the inherent difficulty of detection for some methods. If we include some smaller contributions, such as from cluster contributions ($\Omega_b^{(cl)} \sim 4 \%$)\citep{2004ApJ...616..643F} , or the circumgalactic medium ($\Omega_b^{(CGM)} \sim 5 \% \pm 3\%$), and account for the possiblity that these techniques probe the same phase space, these still do not account for the entirety of the baryon component known from the CMB. 