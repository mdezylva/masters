\begin{abstract}

Modern cosmology is in a golden age. Access to larger and larger datasets have allowed astronomers to construct and constrain some of the most detailed models of the universe available. However, as more data becomes available, there has been an increase in tension between measurements of fundamental cosmological parameters in different areas of the universe. This opens the door for the possibility that our models are wrong, and for the develpment of new physics. One such tension is the missing baryon problem, a discrepency between the measured amount of ordinary baryonic matter at low ($~z<2$) and high redshifts ($z>2$). At high redshift, sources from the Cosmic Microwave Background (CMB), baryon acoustic oscillations, graviational lensing, and quasar absorption spectrum all place a very firm constraint on the amount of matter that exists in the universe. However, when we search for this proportion in the near universe, we find our estimates missing between $30$ and $50$ percent of the expected matter. By making using of a spectral distortion in the CMB, the thermal \sze (tSZ), which acts as a tracer for the large scale structure of the universe, we construct an algorithm which makes use of both CMB and near universe galaxy catalogues to search for this missing baryon fraction. 
\par This work takes the redMaGiC luminous red galaxy catalogue from the \emph{Dark Energy Survey}, and high resolution images of the CMB from the \emph{South Pole Telescope} and the \emph{Planck} satellite, and searches for the correlation between these galaxies and the tSZ signal. A set of 787,058 galaxy pairs were constructed, with line of sight separations $<\SI{20}{\per\h\mega\parsec}$ and transverse separations $\SI{4}{\per\h\mega\parsec} - \SI{20}{\per\h\mega\parsec}$. 
\par We stack pairs by rescaling and rotating the positions of the galaxies so that they sit within a common reference frame. A circularly symmetric halo model is then subtracted from the stacked data in order to obtain a residual signal between the galaxies. 

\par This revealed a $2.05\sigma$ detection of the existence of the missing baryons in a filamentary structure, with a mean Compton-$y$ measurement of $1.29 \times 10^{-8}$. This is in agreement with other measurements of the missing baryon component, and opens the possibility for future work in constraining the missing baryon component at low redshift.

\end{abstract}