\chapter{Introduction}

\section{Motivation}
The Cosmic Microwave Background (CMB) provides the most accurate and detailed measures of the primary cosmological parameters to date. For a $\Lambda$ CDM universe, there are six independant paramters which descirbe the evolution and behaviour of the universe, the physical baryon density $\Omega_b h^2$, the physical dark matter density $\Omega_c h^2$, the age of the universe $t_0$ (or its reciprocal, the Hubble constant $H_0$), the scalar spectral index $n_s$, the curvature fluctuation amplitude $\Delta_R^2$, and the reionisation optical depth $\tau$. 

Currently, the highest precision measures of these features from the CMB come from \cite{2018arXiv180706209P}, which details that baryonic matter only comprises $\approx 5 \% $ of the universe's energy density. In principle, this component of the universe should be directly measurable. At just three minutes after the Big Bang, deuterium can be used as a tracer for this abundance \citep{2007ARNPS..57..463S}, and at redshift $z \geqslant 2$, the baryon fraction can be found in the absorption lines of quasars passing through the diffuse, photo-ionised intergalactic medium, known as the Lyman-$\alpha$ forest \citep{1997ApJ...490..564W}. However as the universe evolved, this gas became sparser as it became more ionised. This makes searching for the entirety of the baryon fraction at low redshift difficult. When this fraction is calculated directly from observations, it shows only one tenth of the baryonic content shown in high redshift measurements is contained in galactic structures \citep{1992MNRAS.258P..14P}. Some revised estimates considered that the limitations of observations were primarily to blame for this discrepency, and not inherent new physics \citep{1994MNRAS.267...13B, 1998ApJ...503..518F}

The baryon content has been confirmed to a very high accuracy with recent CMB experiments, first with the \textit{Wilkinson Microwave Anisotropy Probe} (WMAP) \citep{2007ApJS..170..377S}, and then with the \textit{Planck} Satellite \citep{2018arXiv180706209P}. When we quote quantities, we take values from the latest \textit{Planck} paper


\begin{center}\label{table:params}
 \begin{tabular}{||c c c||} 
 \hline
 Parameter & Value & Error \\
 \hline\hline
 $\Omega_c h^2$ & $0.120$ & $\pm 0.001$ \\
 \hline
 $\Omega_b h^2$ & $0.0224$ & $\pm 0.0001$ \\
 \hline
  $n_s$ & $0.965$ & $\pm 0.004$ \\
 \hline
  $\tau$  & $0.054$ & $\pm 0.007$ \\
 \hline
  $100 \Theta_\star$ & $1.0411$ & $\pm 0.0003$ \\
 \hline
 $H_0$ (km s$^{-1}$ Mpc$^{-1}$) & $67.4$ & $\pm 0.5$ \\
 \hline
\end{tabular}
\end{center}

\subsection{Optical Searches for Baryons}

\subsection{Warm-Hot Intergalactic Medium}


\section{Sunayev-Zeldovich Effect}


