\chapter{The Warm-Hot Interstellar Medium}
High resolution hydrodynamical simulations allow us to predict the overall structure of the cold dark matter in the universe \citep{1999ApJ...514....1C}. These can in turn be used to estimate the baryon distribution at low and moderate redshifts. 

\par It is clear that by the current era, heirachical structure formation collects baryons in gravitational potential wells formed by the dark matter, which moves a significant portion of the baryon component that was previously located in the intergalactic medium at higher redshifts, into structure, such as stars, galaxies, groups, and clusters. 

\par These simulations indicate that the baryons at low redshift fall into four general phases, defined by the overdensity $\delta \equiv \rho/\bar{\rho} - 1$ (where $\bar{\rho}$ is the mean density of baryons). 

\begin{itemize}
\item Diffuse Gas: $\delta <1000$, $T < 10^5 K$, Photoionised gas which is visible in Lyman-$\alpha$ absorption spectra
\item Condensed: $\delta > 1000$ , $T < 10^5 K$, Stars and cool galactic gas
\item Hot: $T > 10^7 K$, Galaxy Clusters and Groups
\item Warm-Hot: $10^5 K < T <10^7 K $, Warm-Hot Intergalactic Medium (WHIM)
\end{itemize}

Simulations \citep{1999ApJ...514....1C,2001ApJ...552..473D} indicate that at redshift $z=0$ approximately $30-40\%$ of baryonic mass is contained within the last catergory, in the WHIM. WHIM gas seems to primarily trace filamentary large scale structures, and clusters around sites of galaxy formation. Because the gas is not bound or virialised, it is apparent that the mechanism which heats it to such high tempratures is shock-heating, caused by gas acreting onto large scale structure. This is consistent with measurements from the soft X-ray background.

\par Because the temperature and density of the WHIM are correlated, and the WHIM is in turn correlated with the large scale structure, we can use the precense of other tracers of large structure, temperature, and density to search for the baryons contained in the WHIM.

\par The WHIM is considered to exist in a filamentary web, which has been shock-heated during the process of structure formation to the temperatures described above. It is so highly ionised, and so disperse (with average densities of the order of 10 particles per cubic meter), that they can only emit or absorb far-ultraviolet or soft x-ray photons. These photons are primarily at highly ionised lines of C, O, Ne, and Fe \citep{2006ApJ...650..573C}.

\par Tracking the baryons contained in the WHIM can only be done by exploiting both experimental multiwavelength observations and theoretical calculations. X-Ray and UV spectroscopic surveys measure the mass of WHIM using the relative and absolute metal content, and the ionisation correction. These can be then combined with optical and infrared photometry and spectroscopy which measure dark matter concentrations by measuring galaxy density around WHIM filaments. These observations then feed into simulations, which allow for more detailed study of virialised structure and the intergalactic medium.

\par The intensity of the signals obtained from direct observation is low in both the UV and the X-Ray bands, both as a result of the density and the relatively small size of the filaments (1 - 10 Mpc). Direct detection ideally requires large field of view and effective area imager-spectrometers, which is not currently available. The strategy that can best be utilised with current technology involves searching for discrete absorption lines in the spectra of bright, featureless background astrophysical sources. 

\par The key feature necessary to detect an absorption line is the ratio between the line wavelength and its equivalent width, called its transition contrast. For the dominant absorption lines of oxygen, in this case OVI, the current resolving power of UV spectrometers is sufficient to measure its transition contrast, but for the X-Ray band, it is worse, and so searches for the WHIM have proven more fruitful in the UV band than in the X-Ray \citep{2005ApJ...624..555D,2006A&A...445..827R}. Using hydrodynamical simulations to replicate the observed absorption per unit redshift, it can be shown that if the WHIM was based solely on the OVI absorption, it would only account for approximately 10 percent of the missing baryon component. 

\par An alternative method for searching for the missing mass is to look for hydrogen absorption in broad Ly-$\alpha$ absorbers (BLAs). At the temperatures that the WHIM is thought to exist at, most of the hydrogen will be ionised, but left-over neutral hydrogen can still imprint Lyman series absorption onto the ultraviolet spectra of background objects. These lines will be be very broad, given that the temperatures of the WHIM create a Doppler parameter of the order of $b\approx \SI{40}{\kilo\meter\per\sec}$. This technique again gives  a similar measurement to that done with OVI absorbers, and so suggests that BLAs and OVI absorbers can be considered to be good tracers of the WHIM, but aren't sufficient to probe the entirety of the missing mass due to the majority of it existing in temperatures only probed by the X-Ray band. 

\par Comprehensive studies of the WHIM therefore require both the UV and X-Ray bands, since the X-Ray is crucial to detect the WHIM, and provide an accurate ionisation correction, and the UV is necessary to measure the assosciated amount of HI and hence the baryonic mass of the system. 

\par According to theory, the chances of finding a WHIM filament along an arbitrary line of sight increases with the path length crossed between the observer and the beacon used to obtain the X-Ray images of intervening space, and the inverse of the baryon column density in the filament in question. This tells us that the larger the amount of baryons in the filament, the lower the probability of finding one. It can be shown that the detection of the WHIM is within the range of instrumentation currently, but it requires long observation times, making it untenable. Searching for an alternative tracer for the WHIM is therefore necessary to accurately locate the missing baryon content.


