\chapter{The Warm-Hot Interstellar Medium}
High resolution hydrodynamical simulations allow us to predict the overall structure of the cold dark matter in the universe \citep{1999ApJ...514....1C}. These can in turn be used to estimate the baryon distribution at low and moderate redshifts. 
\par It is clear that by the current era, heirachical structure formation collects baryons in gravitational potential wells formed by the dark matter, which moves a significant portion of the baryon component that was previously located in the intergalactic medium at higher redshifts, into structure, such as stars, galaxies, groups, and clusters. 
\par These simulations indicate that the baryons at low redshift fall into four general phases, defined by the overdensity $\delta \equiv \rho/\bar{\rho} - 1$ (where $\bar{\rho}$ is the mean density of baryons). 
\begin{itemize}
\item Diffuse Gas: $\delta <1000$, $T < 10^5 K$, Photoionised gas which is visible in Lyman-$\alpha$ absorption spectra
\item Condensed: $\delta > 1000$ , $T < 10^5 K$, Stars and cool galactic gas
\item Hot: $T > 10^7 K$, Galaxy Clusters and Groups
\item Warm-Hot: $10^5 K < T <10^7 K $, Warm-Hot Intergalactic Medium (WHIM)
\end{itemize}
Simulations \citep{1999ApJ...514....1C,2001ApJ...552..473D} indicate that at redshift $z=0$ approximately $30-40\%$ of baryonic mass is contained within the last catergory, in the WHIM. WHIM gas seems to primarily trace filamentary large scale structures, and clusters around sites of galaxy formation. Because the gas is not bound or virialised, it is apparent that the mechanism which heats it to such high tempratures is shock-heating, caused by gas acreting onto large scale structure. This is consistent with measurements from the soft X-ray background.
\par Because the temperature and density of the WHIM are correlated, and the WHIM is in turn correlated with the large scale structure, we can use the precense of other tracers of large structure, temperature, and density to search for the baryons contained in the WHIM.