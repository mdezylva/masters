% Chapter 1

l\chapter{Introduction} % Main chapter title

\label{Introduction} % For referencing the chapter elsewhere, use \ref{Chapter1} 

%----------------------------------------------------------------------------------------

% Define some commands to keep the formatting separated from the content 
\newcommand{\keyword}[1]{\textbf{#1}}
\newcommand{\tabhead}[1]{\textbf{#1}}
\newcommand{\code}[1]{\texttt{#1}}
\newcommand{\file}[1]{\texttt{\bfseries#1}}
\newcommand{\option}[1]{\texttt{\itshape#1}}

%----------------------------------------------------------------------------------------

\section{Modern Cosmology}
 
\subsection{The Big Bang Theory}
The basis for modern cosmology relies on several fundamental assumptions stemming from observation, the chief of which is the Big Bang Model. Following Hubble's discovery of a relation between distances to galaxies and their recessional velocities, the \emph{Copernican Principle} leads to the conclusion that in the past, objects in the universe were much closer together. His observations gave rise to the Lemaitre's Hubble Law, 
\begin{equation}
v  \varpropto d 
\label{eq:HubbleLaw}
\end{equation}
This suggests that at some point in the past, the universe was much smaller than it is at present, the conservation of energy then implies that at some point in the past, the universe must have been an incredibly hot, dense environment. Using general relativity, the extrapolation backwards in time yields a singularity of infinite density and temperature, which is commonly called the \emph{Big Bang}
\par Another assumption stemming from observation is that of isotropy. Based on observation, there appears to be no favoured direction in the universe, since distributions of distant galaxies and other extragalactic sources seem to be evenly distributed across the sky. Perhaps the most spectacular example of this isotropy is the presence of the \emph{Cosmic Microwave Background}. 
\par Discovered in 1964 \citep{Penzias:65}, it was noticed that there was isotropic black-body radiation at $T \approx \SI{2.7}{\kelvin}$. Since the peak of this radiation is in the microwave section of the electromagnetic spectrum, it was termed the \emph{Cosmic Microwave Background}. 
\begin{figure}[ht]
\centering
\includegraphics[scale=0.25]{../Images/CMB_smica_tsig.png} 
\label{CMB Map}
\caption{\emph{Planck} Satellite Full Sky CMB Map}
\end{figure}
This reflects the Big Bang Model, which suggests that space is filled with radiation left over from the initial singularity, and is reinforced by the fact that the background light has a flux orders of magnitude greater than other emitting sources.
\par This picture of the Big Bang, whilst useful at a basic level, presents problems when examined more closely. As it stands, the edges of the observable universe at present are too far away from each other to be causally connected. What accounts for the observed homogeneity and isotropy then, if entire areas of the universe cannot affect each other, or communicate information, and so cannot mix? Measurements of the CMB also indicate that the universe is spatially flat, but the initial conditions required to maintain this state are incredibly specific. There is no reason to suggest that these conditions should be preferentially selected over any others. If the conditions for homogeneity and isotropy are maintained sufficiently to allow this, how then does the observed large scale structure of the universe come about? There are also addition problems arising from particle physics, including considerations regarding magnetic monopoles and gravitinos, which have to be satisfied.

\subsection{Inflation}
The most commonly held solution to these problems is the theory of \emph{cosmological inflation}\cite{Linde:07}. This is a period at a very early epoch of the universe where the expansion rate of the universe is exponentially large. This expansion allows to patches of the universe to be causally connected to each other at early times, as well as flatten out as it evolves, effectively solving the issues of precise tuning with one soluton.

Inflation is driven by an energy field, known as the \emph{inflaton field}. When we treat this field from a quantum mechanical perspective, we observe little lumps of uncertainty in a univorm background appear. As the universe expands, these pockets of uncertainty go from being virtual and mathematical to real and physical. As it expands further, these grow into classical objects, such as galaxies, stars, and the other hallmarks of anisotropy we observe. 

This expansion allows for random fluctiations origininating on the quantum scale to grow into macroscopic ones, whilst still maintain their causal connection at very early times. It therefore allows us to explain large scale regularity and small scale irregularity with the same model. 

The simplest mechanism proposed invokes a scalar field, which we loosely couple to gravity using the equation 
\begin{equation}\label{eq:sclr_inf}
\ddot{\phi} + 3 H \dot{\phi} + V^\prime(\phi) = 0
\end{equation}

The conditions for inflation arise from the Friedmann Equations, and the requirement that the fluctuations grow more rapidly than the characteristic scale on which events are causally connected. 

This manifests as an explcit relation between the pressure gradient of the universe, and its density
\begin{equation}\label{eq:inf_cndtn}
3 P < \rho
\end{equation}
which in turn is satisfied by a \emph{slow-roll} field, whereby 

\begin{equation}\label{eq:slw_rll}
\dot{\phi}^2 < V
\end{equation}
Practically, all the proposed models satisfy three conditions \cite{LiddleLyth:00}. Firstly, the 'force' of the time-varying potential $V^\prime$ balances against the frictive term $3 H \dot{phi}$, which makes the field's motion overdamped. This corresponds to the mathematical relation
\begin{equation}\label{eq:slw_rll_cond1}
\dot{\phi} \simeq - \frac{1}{3 H} V^\prime
\end{equation} 
Secondly, in order to satisfy \ref{eq:slw_rll}, we introduce a parameter $\epsilon$ 
\begin{equation}\label{eq:slw_rll_cond2}
\epsilon \equiv \frac{m_{Pl}^2}{16 \pi} \left(\frac{V^\prime}{V}\right)^2 << 1
\end{equation}
This also leads to the following expression
$$ H^2 \simeq \frac{1}{3} \frac{8 \pi}{m_{Pl}^2} V $$
These together imply that the scale factor, the measure of the rate of expansion of the universe, grows approximately exponentially
$$a \propto e^{Ht} $$
The third condition can be derived from the other two, by differentiation the approximation \ref{eq:slw_rll_cond1}, and maintaining consistency with \ref{eq:sclr_inf}. This third condition is technically independant from the other two however, since there is no requirement that the derivative of an approximation to itself be a valid approximation. 
\begin{equation}\label{eq:slw_rll_cond3}
|\eta| \equiv \left| \frac{m_{Pl}^2}{8 \pi} \frac{V^{\prime \prime}}{V} \right| << 1 
\end{equation}
These conditions are necessary, but are not sufficient, since they only specify conditions of the inflation field, not the form of the field itself, due to the field being subject to a second order scalar wave equation. 

\subsection{Inflation and Gravitational Waves}
Regardless of the precise model of inflation, all models predict that the period of rapid expansion would have produced primordial gravitational waves, being produced as a vacuum fluctuation in the same way as the density fluctuations. 

\par In the Friedmann-Robertson-Walker Universe, a gravitational wave corresponds to a spatial metric perturbation. 


Inflation is a powerful tool to solve theoretical problems, however we lack direct confirmation of its mechanism. Particle physics is currently agnostic on direct detection of particles created by the field, and so we must search for tracers elsewhere. 






